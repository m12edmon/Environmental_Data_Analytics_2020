\PassOptionsToPackage{unicode=true}{hyperref} % options for packages loaded elsewhere
\PassOptionsToPackage{hyphens}{url}
%
\documentclass[]{article}
\usepackage{lmodern}
\usepackage{amssymb,amsmath}
\usepackage{ifxetex,ifluatex}
\usepackage{fixltx2e} % provides \textsubscript
\ifnum 0\ifxetex 1\fi\ifluatex 1\fi=0 % if pdftex
  \usepackage[T1]{fontenc}
  \usepackage[utf8]{inputenc}
  \usepackage{textcomp} % provides euro and other symbols
\else % if luatex or xelatex
  \usepackage{unicode-math}
  \defaultfontfeatures{Ligatures=TeX,Scale=MatchLowercase}
\fi
% use upquote if available, for straight quotes in verbatim environments
\IfFileExists{upquote.sty}{\usepackage{upquote}}{}
% use microtype if available
\IfFileExists{microtype.sty}{%
\usepackage[]{microtype}
\UseMicrotypeSet[protrusion]{basicmath} % disable protrusion for tt fonts
}{}
\IfFileExists{parskip.sty}{%
\usepackage{parskip}
}{% else
\setlength{\parindent}{0pt}
\setlength{\parskip}{6pt plus 2pt minus 1pt}
}
\usepackage{hyperref}
\hypersetup{
            pdftitle={Assignment 4: Data Wrangling},
            pdfauthor={Masha Edmondson},
            pdfborder={0 0 0},
            breaklinks=true}
\urlstyle{same}  % don't use monospace font for urls
\usepackage[margin=2.54cm]{geometry}
\usepackage{color}
\usepackage{fancyvrb}
\newcommand{\VerbBar}{|}
\newcommand{\VERB}{\Verb[commandchars=\\\{\}]}
\DefineVerbatimEnvironment{Highlighting}{Verbatim}{commandchars=\\\{\}}
% Add ',fontsize=\small' for more characters per line
\usepackage{framed}
\definecolor{shadecolor}{RGB}{248,248,248}
\newenvironment{Shaded}{\begin{snugshade}}{\end{snugshade}}
\newcommand{\AlertTok}[1]{\textcolor[rgb]{0.94,0.16,0.16}{#1}}
\newcommand{\AnnotationTok}[1]{\textcolor[rgb]{0.56,0.35,0.01}{\textbf{\textit{#1}}}}
\newcommand{\AttributeTok}[1]{\textcolor[rgb]{0.77,0.63,0.00}{#1}}
\newcommand{\BaseNTok}[1]{\textcolor[rgb]{0.00,0.00,0.81}{#1}}
\newcommand{\BuiltInTok}[1]{#1}
\newcommand{\CharTok}[1]{\textcolor[rgb]{0.31,0.60,0.02}{#1}}
\newcommand{\CommentTok}[1]{\textcolor[rgb]{0.56,0.35,0.01}{\textit{#1}}}
\newcommand{\CommentVarTok}[1]{\textcolor[rgb]{0.56,0.35,0.01}{\textbf{\textit{#1}}}}
\newcommand{\ConstantTok}[1]{\textcolor[rgb]{0.00,0.00,0.00}{#1}}
\newcommand{\ControlFlowTok}[1]{\textcolor[rgb]{0.13,0.29,0.53}{\textbf{#1}}}
\newcommand{\DataTypeTok}[1]{\textcolor[rgb]{0.13,0.29,0.53}{#1}}
\newcommand{\DecValTok}[1]{\textcolor[rgb]{0.00,0.00,0.81}{#1}}
\newcommand{\DocumentationTok}[1]{\textcolor[rgb]{0.56,0.35,0.01}{\textbf{\textit{#1}}}}
\newcommand{\ErrorTok}[1]{\textcolor[rgb]{0.64,0.00,0.00}{\textbf{#1}}}
\newcommand{\ExtensionTok}[1]{#1}
\newcommand{\FloatTok}[1]{\textcolor[rgb]{0.00,0.00,0.81}{#1}}
\newcommand{\FunctionTok}[1]{\textcolor[rgb]{0.00,0.00,0.00}{#1}}
\newcommand{\ImportTok}[1]{#1}
\newcommand{\InformationTok}[1]{\textcolor[rgb]{0.56,0.35,0.01}{\textbf{\textit{#1}}}}
\newcommand{\KeywordTok}[1]{\textcolor[rgb]{0.13,0.29,0.53}{\textbf{#1}}}
\newcommand{\NormalTok}[1]{#1}
\newcommand{\OperatorTok}[1]{\textcolor[rgb]{0.81,0.36,0.00}{\textbf{#1}}}
\newcommand{\OtherTok}[1]{\textcolor[rgb]{0.56,0.35,0.01}{#1}}
\newcommand{\PreprocessorTok}[1]{\textcolor[rgb]{0.56,0.35,0.01}{\textit{#1}}}
\newcommand{\RegionMarkerTok}[1]{#1}
\newcommand{\SpecialCharTok}[1]{\textcolor[rgb]{0.00,0.00,0.00}{#1}}
\newcommand{\SpecialStringTok}[1]{\textcolor[rgb]{0.31,0.60,0.02}{#1}}
\newcommand{\StringTok}[1]{\textcolor[rgb]{0.31,0.60,0.02}{#1}}
\newcommand{\VariableTok}[1]{\textcolor[rgb]{0.00,0.00,0.00}{#1}}
\newcommand{\VerbatimStringTok}[1]{\textcolor[rgb]{0.31,0.60,0.02}{#1}}
\newcommand{\WarningTok}[1]{\textcolor[rgb]{0.56,0.35,0.01}{\textbf{\textit{#1}}}}
\usepackage{graphicx,grffile}
\makeatletter
\def\maxwidth{\ifdim\Gin@nat@width>\linewidth\linewidth\else\Gin@nat@width\fi}
\def\maxheight{\ifdim\Gin@nat@height>\textheight\textheight\else\Gin@nat@height\fi}
\makeatother
% Scale images if necessary, so that they will not overflow the page
% margins by default, and it is still possible to overwrite the defaults
% using explicit options in \includegraphics[width, height, ...]{}
\setkeys{Gin}{width=\maxwidth,height=\maxheight,keepaspectratio}
\setlength{\emergencystretch}{3em}  % prevent overfull lines
\providecommand{\tightlist}{%
  \setlength{\itemsep}{0pt}\setlength{\parskip}{0pt}}
\setcounter{secnumdepth}{0}
% Redefines (sub)paragraphs to behave more like sections
\ifx\paragraph\undefined\else
\let\oldparagraph\paragraph
\renewcommand{\paragraph}[1]{\oldparagraph{#1}\mbox{}}
\fi
\ifx\subparagraph\undefined\else
\let\oldsubparagraph\subparagraph
\renewcommand{\subparagraph}[1]{\oldsubparagraph{#1}\mbox{}}
\fi

% set default figure placement to htbp
\makeatletter
\def\fps@figure{htbp}
\makeatother


\title{Assignment 4: Data Wrangling}
\author{Masha Edmondson}
\date{}

\begin{document}
\maketitle

\hypertarget{overview}{%
\subsection{OVERVIEW}\label{overview}}

This exercise accompanies the lessons in Environmental Data Analytics on
Data Wrangling

\hypertarget{directions}{%
\subsection{Directions}\label{directions}}

\begin{enumerate}
\def\labelenumi{\arabic{enumi}.}
\tightlist
\item
  Change ``Student Name'' on line 3 (above) with your name.
\item
  Work through the steps, \textbf{creating code and output} that fulfill
  each instruction.
\item
  Be sure to \textbf{answer the questions} in this assignment document.
\item
  When you have completed the assignment, \textbf{Knit} the text and
  code into a single PDF file.
\item
  After Knitting, submit the completed exercise (PDF file) to the
  dropbox in Sakai. Add your last name into the file name (e.g.,
  ``Salk\_A04\_DataWrangling.Rmd'') prior to submission.
\end{enumerate}

The completed exercise is due on Tuesday, February 4 at 1:00 pm.

\hypertarget{set-up-your-session}{%
\subsection{Set up your session}\label{set-up-your-session}}

\begin{enumerate}
\def\labelenumi{\arabic{enumi}.}
\tightlist
\item
  Check your working directory, load the \texttt{tidyverse} and
  \texttt{lubridate} packages, and upload all four raw data files
  associated with the EPA Air dataset. See the README file for the EPA
  air datasets for more information (especially if you have not worked
  with air quality data previously).
\end{enumerate}

\begin{Shaded}
\begin{Highlighting}[]
\CommentTok{# 1. Set up your working directory}
\KeywordTok{getwd}\NormalTok{()}
\end{Highlighting}
\end{Shaded}

\begin{verbatim}
## [1] "/Users/mashaedmondson/Desktop/Environmental_Data_Analytics_2020"
\end{verbatim}

\begin{Shaded}
\begin{Highlighting}[]
\CommentTok{# 2. Load packges}
\KeywordTok{library}\NormalTok{(tidyverse)}
\CommentTok{#install.packages(lubridate)}
\KeywordTok{library}\NormalTok{(lubridate)}

\CommentTok{# 3. Import datasets}
\NormalTok{EPAair_O3_NC2018 <-}\StringTok{ }\KeywordTok{read.csv}\NormalTok{(}\StringTok{"./Data/Raw/EPAair_O3_NC2018_raw.csv"}\NormalTok{)}
\NormalTok{EPAair_O3_NC2019 <-}\StringTok{ }\KeywordTok{read.csv}\NormalTok{(}\StringTok{"./Data/Raw/EPAair_O3_NC2019_raw.csv"}\NormalTok{)}
\NormalTok{EPAair_PM25_NC2018 <-}\StringTok{ }\KeywordTok{read.csv}\NormalTok{(}\StringTok{"./Data/Raw/EPAair_PM25_NC2018_raw.csv"}\NormalTok{)}
\NormalTok{EPAair_PM25_NC2019 <-}\StringTok{ }\KeywordTok{read.csv}\NormalTok{(}\StringTok{"./Data/Raw/EPAair_PM25_NC2019_raw.csv"}\NormalTok{)}
\end{Highlighting}
\end{Shaded}

\begin{enumerate}
\def\labelenumi{\arabic{enumi}.}
\setcounter{enumi}{1}
\tightlist
\item
  Explore the dimensions, column names, and structure of the datasets.
\end{enumerate}

\begin{Shaded}
\begin{Highlighting}[]
\KeywordTok{colnames}\NormalTok{(EPAair_O3_NC2018)}
\end{Highlighting}
\end{Shaded}

\begin{verbatim}
##  [1] "Date"                                
##  [2] "Source"                              
##  [3] "Site.ID"                             
##  [4] "POC"                                 
##  [5] "Daily.Max.8.hour.Ozone.Concentration"
##  [6] "UNITS"                               
##  [7] "DAILY_AQI_VALUE"                     
##  [8] "Site.Name"                           
##  [9] "DAILY_OBS_COUNT"                     
## [10] "PERCENT_COMPLETE"                    
## [11] "AQS_PARAMETER_CODE"                  
## [12] "AQS_PARAMETER_DESC"                  
## [13] "CBSA_CODE"                           
## [14] "CBSA_NAME"                           
## [15] "STATE_CODE"                          
## [16] "STATE"                               
## [17] "COUNTY_CODE"                         
## [18] "COUNTY"                              
## [19] "SITE_LATITUDE"                       
## [20] "SITE_LONGITUDE"
\end{verbatim}

\begin{Shaded}
\begin{Highlighting}[]
\KeywordTok{dim}\NormalTok{(EPAair_O3_NC2018)}
\end{Highlighting}
\end{Shaded}

\begin{verbatim}
## [1] 9737   20
\end{verbatim}

\begin{Shaded}
\begin{Highlighting}[]
\KeywordTok{str}\NormalTok{(EPAair_O3_NC2018)}
\end{Highlighting}
\end{Shaded}

\begin{verbatim}
## 'data.frame':    9737 obs. of  20 variables:
##  $ Date                                : Factor w/ 364 levels "01/01/2018","01/02/2018",..: 60 61 62 63 64 65 66 67 68 69 ...
##  $ Source                              : Factor w/ 1 level "AQS": 1 1 1 1 1 1 1 1 1 1 ...
##  $ Site.ID                             : int  370030005 370030005 370030005 370030005 370030005 370030005 370030005 370030005 370030005 370030005 ...
##  $ POC                                 : int  1 1 1 1 1 1 1 1 1 1 ...
##  $ Daily.Max.8.hour.Ozone.Concentration: num  0.043 0.046 0.047 0.049 0.047 0.03 0.036 0.044 0.049 0.043 ...
##  $ UNITS                               : Factor w/ 1 level "ppm": 1 1 1 1 1 1 1 1 1 1 ...
##  $ DAILY_AQI_VALUE                     : int  40 43 44 45 44 28 33 41 45 40 ...
##  $ Site.Name                           : Factor w/ 40 levels "","Beaufort",..: 35 35 35 35 35 35 35 35 35 35 ...
##  $ DAILY_OBS_COUNT                     : int  17 17 17 17 17 17 17 17 17 17 ...
##  $ PERCENT_COMPLETE                    : num  100 100 100 100 100 100 100 100 100 100 ...
##  $ AQS_PARAMETER_CODE                  : int  44201 44201 44201 44201 44201 44201 44201 44201 44201 44201 ...
##  $ AQS_PARAMETER_DESC                  : Factor w/ 1 level "Ozone": 1 1 1 1 1 1 1 1 1 1 ...
##  $ CBSA_CODE                           : int  25860 25860 25860 25860 25860 25860 25860 25860 25860 25860 ...
##  $ CBSA_NAME                           : Factor w/ 17 levels "","Asheville, NC",..: 9 9 9 9 9 9 9 9 9 9 ...
##  $ STATE_CODE                          : int  37 37 37 37 37 37 37 37 37 37 ...
##  $ STATE                               : Factor w/ 1 level "North Carolina": 1 1 1 1 1 1 1 1 1 1 ...
##  $ COUNTY_CODE                         : int  3 3 3 3 3 3 3 3 3 3 ...
##  $ COUNTY                              : Factor w/ 32 levels "Alexander","Avery",..: 1 1 1 1 1 1 1 1 1 1 ...
##  $ SITE_LATITUDE                       : num  35.9 35.9 35.9 35.9 35.9 ...
##  $ SITE_LONGITUDE                      : num  -81.2 -81.2 -81.2 -81.2 -81.2 ...
\end{verbatim}

\begin{Shaded}
\begin{Highlighting}[]
\KeywordTok{summary}\NormalTok{(EPAair_O3_NC2018)}
\end{Highlighting}
\end{Shaded}

\begin{verbatim}
##          Date      Source        Site.ID               POC   
##  04/01/2018:  40   AQS:9737   Min.   :370030005   Min.   :1  
##  04/12/2018:  40              1st Qu.:370650099   1st Qu.:1  
##  04/13/2018:  40              Median :371010002   Median :1  
##  04/14/2018:  40              Mean   :370969118   Mean   :1  
##  04/15/2018:  40              3rd Qu.:371290002   3rd Qu.:1  
##  04/18/2018:  40              Max.   :371990004   Max.   :1  
##  (Other)   :9497                                             
##  Daily.Max.8.hour.Ozone.Concentration UNITS      DAILY_AQI_VALUE 
##  Min.   :0.00200                      ppm:9737   Min.   :  2.00  
##  1st Qu.:0.03400                                 1st Qu.: 31.00  
##  Median :0.04200                                 Median : 39.00  
##  Mean   :0.04194                                 Mean   : 40.22  
##  3rd Qu.:0.04900                                 3rd Qu.: 45.00  
##  Max.   :0.07700                                 Max.   :122.00  
##                                                                  
##                 Site.Name    DAILY_OBS_COUNT PERCENT_COMPLETE
##  Coweeta             : 355   Min.   :12.00   Min.   : 71.00  
##  Garinger High School: 354   1st Qu.:17.00   1st Qu.:100.00  
##  Millbrook School    : 352   Median :17.00   Median :100.00  
##  Candor              : 335   Mean   :16.94   Mean   : 99.65  
##  Rockwell            : 335   3rd Qu.:17.00   3rd Qu.:100.00  
##  Cranberry           : 323   Max.   :17.00   Max.   :100.00  
##  (Other)             :7683                                   
##  AQS_PARAMETER_CODE AQS_PARAMETER_DESC   CBSA_CODE    
##  Min.   :44201      Ozone:9737         Min.   :11700  
##  1st Qu.:44201                         1st Qu.:16740  
##  Median :44201                         Median :24660  
##  Mean   :44201                         Mean   :27247  
##  3rd Qu.:44201                         3rd Qu.:39580  
##  Max.   :44201                         Max.   :49180  
##                                        NA's   :2609   
##                              CBSA_NAME      STATE_CODE            STATE     
##                                   :2609   Min.   :37   North Carolina:9737  
##  Charlotte-Concord-Gastonia, NC-SC:1338   1st Qu.:37                        
##  Asheville, NC                    : 927   Median :37                        
##  Winston-Salem, NC                : 725   Mean   :37                        
##  Raleigh, NC                      : 585   3rd Qu.:37                        
##  Hickory-Lenoir-Morganton, NC     : 477   Max.   :37                        
##  (Other)                          :3076                                     
##   COUNTY_CODE             COUNTY     SITE_LATITUDE   SITE_LONGITUDE  
##  Min.   :  3.00   Forsyth    : 725   Min.   :34.36   Min.   :-83.80  
##  1st Qu.: 65.00   Haywood    : 683   1st Qu.:35.26   1st Qu.:-82.05  
##  Median :101.00   Mecklenburg: 592   Median :35.55   Median :-80.34  
##  Mean   : 96.78   Avery      : 558   Mean   :35.62   Mean   :-80.42  
##  3rd Qu.:129.00   Swain      : 483   3rd Qu.:36.03   3rd Qu.:-78.90  
##  Max.   :199.00   Cumberland : 444   Max.   :36.31   Max.   :-76.62  
##                   (Other)    :6252
\end{verbatim}

\begin{Shaded}
\begin{Highlighting}[]
\KeywordTok{class}\NormalTok{(EPAair_O3_NC2018)}
\end{Highlighting}
\end{Shaded}

\begin{verbatim}
## [1] "data.frame"
\end{verbatim}

\begin{Shaded}
\begin{Highlighting}[]
\KeywordTok{colnames}\NormalTok{(EPAair_O3_NC2019)}
\end{Highlighting}
\end{Shaded}

\begin{verbatim}
##  [1] "Date"                                
##  [2] "Source"                              
##  [3] "Site.ID"                             
##  [4] "POC"                                 
##  [5] "Daily.Max.8.hour.Ozone.Concentration"
##  [6] "UNITS"                               
##  [7] "DAILY_AQI_VALUE"                     
##  [8] "Site.Name"                           
##  [9] "DAILY_OBS_COUNT"                     
## [10] "PERCENT_COMPLETE"                    
## [11] "AQS_PARAMETER_CODE"                  
## [12] "AQS_PARAMETER_DESC"                  
## [13] "CBSA_CODE"                           
## [14] "CBSA_NAME"                           
## [15] "STATE_CODE"                          
## [16] "STATE"                               
## [17] "COUNTY_CODE"                         
## [18] "COUNTY"                              
## [19] "SITE_LATITUDE"                       
## [20] "SITE_LONGITUDE"
\end{verbatim}

\begin{Shaded}
\begin{Highlighting}[]
\KeywordTok{dim}\NormalTok{(EPAair_O3_NC2019)}
\end{Highlighting}
\end{Shaded}

\begin{verbatim}
## [1] 10592    20
\end{verbatim}

\begin{Shaded}
\begin{Highlighting}[]
\KeywordTok{str}\NormalTok{(EPAair_O3_NC2019)}
\end{Highlighting}
\end{Shaded}

\begin{verbatim}
## 'data.frame':    10592 obs. of  20 variables:
##  $ Date                                : Factor w/ 365 levels "01/01/2019","01/02/2019",..: 1 2 3 4 5 6 7 8 9 10 ...
##  $ Source                              : Factor w/ 2 levels "AirNow","AQS": 1 1 1 1 1 1 1 1 1 1 ...
##  $ Site.ID                             : int  370030005 370030005 370030005 370030005 370030005 370030005 370030005 370030005 370030005 370030005 ...
##  $ POC                                 : int  1 1 1 1 1 1 1 1 1 1 ...
##  $ Daily.Max.8.hour.Ozone.Concentration: num  0.029 0.018 0.016 0.022 0.037 0.037 0.029 0.038 0.038 0.03 ...
##  $ UNITS                               : Factor w/ 1 level "ppm": 1 1 1 1 1 1 1 1 1 1 ...
##  $ DAILY_AQI_VALUE                     : int  27 17 15 20 34 34 27 35 35 28 ...
##  $ Site.Name                           : Factor w/ 38 levels "","Beaufort",..: 33 33 33 33 33 33 33 33 33 33 ...
##  $ DAILY_OBS_COUNT                     : int  24 24 24 24 24 24 24 24 24 24 ...
##  $ PERCENT_COMPLETE                    : num  100 100 100 100 100 100 100 100 100 100 ...
##  $ AQS_PARAMETER_CODE                  : int  44201 44201 44201 44201 44201 44201 44201 44201 44201 44201 ...
##  $ AQS_PARAMETER_DESC                  : Factor w/ 1 level "Ozone": 1 1 1 1 1 1 1 1 1 1 ...
##  $ CBSA_CODE                           : int  25860 25860 25860 25860 25860 25860 25860 25860 25860 25860 ...
##  $ CBSA_NAME                           : Factor w/ 15 levels "","Asheville, NC",..: 8 8 8 8 8 8 8 8 8 8 ...
##  $ STATE_CODE                          : int  37 37 37 37 37 37 37 37 37 37 ...
##  $ STATE                               : Factor w/ 1 level "North Carolina": 1 1 1 1 1 1 1 1 1 1 ...
##  $ COUNTY_CODE                         : int  3 3 3 3 3 3 3 3 3 3 ...
##  $ COUNTY                              : Factor w/ 30 levels "Alexander","Avery",..: 1 1 1 1 1 1 1 1 1 1 ...
##  $ SITE_LATITUDE                       : num  35.9 35.9 35.9 35.9 35.9 ...
##  $ SITE_LONGITUDE                      : num  -81.2 -81.2 -81.2 -81.2 -81.2 ...
\end{verbatim}

\begin{Shaded}
\begin{Highlighting}[]
\KeywordTok{summary}\NormalTok{(EPAair_O3_NC2019)}
\end{Highlighting}
\end{Shaded}

\begin{verbatim}
##          Date          Source        Site.ID               POC   
##  03/18/2019:   38   AirNow:2126   Min.   :370030005   Min.   :1  
##  03/19/2019:   38   AQS   :8466   1st Qu.:370630015   1st Qu.:1  
##  03/20/2019:   38                 Median :370870036   Median :1  
##  03/23/2019:   38                 Mean   :370960317   Mean   :1  
##  03/24/2019:   38                 3rd Qu.:371290002   3rd Qu.:1  
##  03/25/2019:   38                 Max.   :371990004   Max.   :1  
##  (Other)   :10364                                                
##  Daily.Max.8.hour.Ozone.Concentration UNITS       DAILY_AQI_VALUE
##  Min.   :0.00000                      ppm:10592   Min.   :  0.0  
##  1st Qu.:0.03600                                  1st Qu.: 33.0  
##  Median :0.04400                                  Median : 41.0  
##  Mean   :0.04331                                  Mean   : 41.2  
##  3rd Qu.:0.05000                                  3rd Qu.: 46.0  
##  Max.   :0.08100                                  Max.   :136.0  
##                                                                  
##                 Site.Name    DAILY_OBS_COUNT PERCENT_COMPLETE
##  Garinger High School: 363   Min.   :13.00   Min.   : 75.00  
##  Millbrook School    : 362   1st Qu.:17.00   1st Qu.:100.00  
##  Coweeta             : 361   Median :17.00   Median :100.00  
##  Rockwell            : 361   Mean   :18.34   Mean   : 99.69  
##  Candor              : 358   3rd Qu.:17.00   3rd Qu.:100.00  
##  Cranberry           : 351   Max.   :24.00   Max.   :100.00  
##  (Other)             :8436                                   
##  AQS_PARAMETER_CODE AQS_PARAMETER_DESC   CBSA_CODE    
##  Min.   :44201      Ozone:10592        Min.   :11700  
##  1st Qu.:44201                         1st Qu.:16740  
##  Median :44201                         Median :24660  
##  Mean   :44201                         Mean   :26617  
##  3rd Qu.:44201                         3rd Qu.:37080  
##  Max.   :44201                         Max.   :49180  
##                                        NA's   :2852   
##                              CBSA_NAME      STATE_CODE            STATE      
##                                   :2852   Min.   :37   North Carolina:10592  
##  Charlotte-Concord-Gastonia, NC-SC:1590   1st Qu.:37                         
##  Asheville, NC                    :1114   Median :37                         
##  Winston-Salem, NC                : 735   Mean   :37                         
##  Raleigh, NC                      : 646   3rd Qu.:37                         
##  Hickory-Lenoir-Morganton, NC     : 567   Max.   :37                         
##  (Other)                          :3088                                      
##   COUNTY_CODE            COUNTY     SITE_LATITUDE   SITE_LONGITUDE  
##  Min.   :  3.0   Haywood    : 864   Min.   :34.36   Min.   :-83.80  
##  1st Qu.: 63.0   Forsyth    : 735   1st Qu.:35.26   1st Qu.:-82.05  
##  Median : 87.0   Mecklenburg: 657   Median :35.59   Median :-80.34  
##  Mean   : 95.9   Avery      : 607   Mean   :35.61   Mean   :-80.41  
##  3rd Qu.:129.0   Cumberland : 498   3rd Qu.:36.03   3rd Qu.:-78.77  
##  Max.   :199.0   Swain      : 476   Max.   :36.31   Max.   :-76.62  
##                  (Other)    :6755
\end{verbatim}

\begin{Shaded}
\begin{Highlighting}[]
\KeywordTok{class}\NormalTok{(EPAair_O3_NC2019)}
\end{Highlighting}
\end{Shaded}

\begin{verbatim}
## [1] "data.frame"
\end{verbatim}

\begin{Shaded}
\begin{Highlighting}[]
\KeywordTok{colnames}\NormalTok{(EPAair_PM25_NC2018)}
\end{Highlighting}
\end{Shaded}

\begin{verbatim}
##  [1] "Date"                           "Source"                        
##  [3] "Site.ID"                        "POC"                           
##  [5] "Daily.Mean.PM2.5.Concentration" "UNITS"                         
##  [7] "DAILY_AQI_VALUE"                "Site.Name"                     
##  [9] "DAILY_OBS_COUNT"                "PERCENT_COMPLETE"              
## [11] "AQS_PARAMETER_CODE"             "AQS_PARAMETER_DESC"            
## [13] "CBSA_CODE"                      "CBSA_NAME"                     
## [15] "STATE_CODE"                     "STATE"                         
## [17] "COUNTY_CODE"                    "COUNTY"                        
## [19] "SITE_LATITUDE"                  "SITE_LONGITUDE"
\end{verbatim}

\begin{Shaded}
\begin{Highlighting}[]
\KeywordTok{dim}\NormalTok{(EPAair_PM25_NC2018)}
\end{Highlighting}
\end{Shaded}

\begin{verbatim}
## [1] 8983   20
\end{verbatim}

\begin{Shaded}
\begin{Highlighting}[]
\KeywordTok{str}\NormalTok{(EPAair_PM25_NC2018)}
\end{Highlighting}
\end{Shaded}

\begin{verbatim}
## 'data.frame':    8983 obs. of  20 variables:
##  $ Date                          : Factor w/ 365 levels "01/01/2018","01/02/2018",..: 2 5 8 11 14 17 20 23 26 29 ...
##  $ Source                        : Factor w/ 1 level "AQS": 1 1 1 1 1 1 1 1 1 1 ...
##  $ Site.ID                       : int  370110002 370110002 370110002 370110002 370110002 370110002 370110002 370110002 370110002 370110002 ...
##  $ POC                           : int  1 1 1 1 1 1 1 1 1 1 ...
##  $ Daily.Mean.PM2.5.Concentration: num  2.9 3.7 5.3 0.8 2.5 4.5 1.8 2.5 4.2 1.7 ...
##  $ UNITS                         : Factor w/ 1 level "ug/m3 LC": 1 1 1 1 1 1 1 1 1 1 ...
##  $ DAILY_AQI_VALUE               : int  12 15 22 3 10 19 8 10 18 7 ...
##  $ Site.Name                     : Factor w/ 25 levels "","Blackstone",..: 15 15 15 15 15 15 15 15 15 15 ...
##  $ DAILY_OBS_COUNT               : int  1 1 1 1 1 1 1 1 1 1 ...
##  $ PERCENT_COMPLETE              : num  100 100 100 100 100 100 100 100 100 100 ...
##  $ AQS_PARAMETER_CODE            : int  88502 88502 88502 88502 88502 88502 88502 88502 88502 88502 ...
##  $ AQS_PARAMETER_DESC            : Factor w/ 2 levels "Acceptable PM2.5 AQI & Speciation Mass",..: 1 1 1 1 1 1 1 1 1 1 ...
##  $ CBSA_CODE                     : int  NA NA NA NA NA NA NA NA NA NA ...
##  $ CBSA_NAME                     : Factor w/ 14 levels "","Asheville, NC",..: 1 1 1 1 1 1 1 1 1 1 ...
##  $ STATE_CODE                    : int  37 37 37 37 37 37 37 37 37 37 ...
##  $ STATE                         : Factor w/ 1 level "North Carolina": 1 1 1 1 1 1 1 1 1 1 ...
##  $ COUNTY_CODE                   : int  11 11 11 11 11 11 11 11 11 11 ...
##  $ COUNTY                        : Factor w/ 21 levels "Avery","Buncombe",..: 1 1 1 1 1 1 1 1 1 1 ...
##  $ SITE_LATITUDE                 : num  36 36 36 36 36 ...
##  $ SITE_LONGITUDE                : num  -81.9 -81.9 -81.9 -81.9 -81.9 ...
\end{verbatim}

\begin{Shaded}
\begin{Highlighting}[]
\KeywordTok{summary}\NormalTok{(EPAair_PM25_NC2018)}
\end{Highlighting}
\end{Shaded}

\begin{verbatim}
##          Date      Source        Site.ID               POC       
##  01/26/2018:  40   AQS:8983   Min.   :370110002   Min.   :1.000  
##  02/01/2018:  40              1st Qu.:370630015   1st Qu.:3.000  
##  02/19/2018:  40              Median :371010002   Median :3.000  
##  03/21/2018:  40              Mean   :371002405   Mean   :2.812  
##  04/02/2018:  40              3rd Qu.:371230001   3rd Qu.:3.000  
##  04/08/2018:  40              Max.   :371830021   Max.   :5.000  
##  (Other)   :8743                                                 
##  Daily.Mean.PM2.5.Concentration      UNITS      DAILY_AQI_VALUE
##  Min.   :-2.300                 ug/m3 LC:8983   Min.   : 0.00  
##  1st Qu.: 4.900                                 1st Qu.:20.00  
##  Median : 7.000                                 Median :29.00  
##  Mean   : 7.491                                 Mean   :30.73  
##  3rd Qu.: 9.700                                 3rd Qu.:40.00  
##  Max.   :34.200                                 Max.   :97.00  
##                                                                
##                 Site.Name    DAILY_OBS_COUNT PERCENT_COMPLETE
##  Millbrook School    : 717   Min.   :1       Min.   :100     
##  Hattie Avenue       : 510   1st Qu.:1       1st Qu.:100     
##  Board Of Ed. Bldg.  : 477   Median :1       Median :100     
##  Garinger High School: 472   Mean   :1       Mean   :100     
##  Durham Armory       : 466   3rd Qu.:1       3rd Qu.:100     
##  Pitt Agri. Center   : 460   Max.   :1       Max.   :100     
##  (Other)             :5881                                   
##  AQS_PARAMETER_CODE                              AQS_PARAMETER_DESC
##  Min.   :88101      Acceptable PM2.5 AQI & Speciation Mass:1403    
##  1st Qu.:88101      PM2.5 - Local Conditions              :7580    
##  Median :88101                                                     
##  Mean   :88164                                                     
##  3rd Qu.:88101                                                     
##  Max.   :88502                                                     
##                                                                    
##    CBSA_CODE                                 CBSA_NAME      STATE_CODE
##  Min.   :11700   Raleigh, NC                      :1396   Min.   :37  
##  1st Qu.:19000   Winston-Salem, NC                :1316   1st Qu.:37  
##  Median :25860   Charlotte-Concord-Gastonia, NC-SC:1275   Median :37  
##  Mean   :30946                                    :1263   Mean   :37  
##  3rd Qu.:40580   Asheville, NC                    : 586   3rd Qu.:37  
##  Max.   :49180   Durham-Chapel Hill, NC           : 466   Max.   :37  
##  NA's   :1263    (Other)                          :2681               
##             STATE       COUNTY_CODE            COUNTY     SITE_LATITUDE  
##  North Carolina:8983   Min.   : 11.0   Mecklenburg:1275   Min.   :34.36  
##                        1st Qu.: 63.0   Wake       :1049   1st Qu.:35.26  
##                        Median :101.0   Forsyth    : 876   Median :35.64  
##                        Mean   :100.2   Buncombe   : 477   Mean   :35.61  
##                        3rd Qu.:123.0   Durham     : 466   3rd Qu.:35.91  
##                        Max.   :183.0   Pitt       : 460   Max.   :36.11  
##                                        (Other)    :4380                  
##  SITE_LONGITUDE  
##  Min.   :-83.44  
##  1st Qu.:-80.87  
##  Median :-80.23  
##  Mean   :-79.99  
##  3rd Qu.:-78.57  
##  Max.   :-76.21  
## 
\end{verbatim}

\begin{Shaded}
\begin{Highlighting}[]
\KeywordTok{class}\NormalTok{(EPAair_PM25_NC2018)}
\end{Highlighting}
\end{Shaded}

\begin{verbatim}
## [1] "data.frame"
\end{verbatim}

\begin{Shaded}
\begin{Highlighting}[]
\KeywordTok{colnames}\NormalTok{(EPAair_PM25_NC2019)}
\end{Highlighting}
\end{Shaded}

\begin{verbatim}
##  [1] "Date"                           "Source"                        
##  [3] "Site.ID"                        "POC"                           
##  [5] "Daily.Mean.PM2.5.Concentration" "UNITS"                         
##  [7] "DAILY_AQI_VALUE"                "Site.Name"                     
##  [9] "DAILY_OBS_COUNT"                "PERCENT_COMPLETE"              
## [11] "AQS_PARAMETER_CODE"             "AQS_PARAMETER_DESC"            
## [13] "CBSA_CODE"                      "CBSA_NAME"                     
## [15] "STATE_CODE"                     "STATE"                         
## [17] "COUNTY_CODE"                    "COUNTY"                        
## [19] "SITE_LATITUDE"                  "SITE_LONGITUDE"
\end{verbatim}

\begin{Shaded}
\begin{Highlighting}[]
\KeywordTok{dim}\NormalTok{(EPAair_PM25_NC2019)}
\end{Highlighting}
\end{Shaded}

\begin{verbatim}
## [1] 8581   20
\end{verbatim}

\begin{Shaded}
\begin{Highlighting}[]
\KeywordTok{str}\NormalTok{(EPAair_PM25_NC2019)}
\end{Highlighting}
\end{Shaded}

\begin{verbatim}
## 'data.frame':    8581 obs. of  20 variables:
##  $ Date                          : Factor w/ 365 levels "01/01/2019","01/02/2019",..: 3 6 9 12 15 18 21 24 27 30 ...
##  $ Source                        : Factor w/ 2 levels "AirNow","AQS": 2 2 2 2 2 2 2 2 2 2 ...
##  $ Site.ID                       : int  370110002 370110002 370110002 370110002 370110002 370110002 370110002 370110002 370110002 370110002 ...
##  $ POC                           : int  1 1 1 1 1 1 1 1 1 1 ...
##  $ Daily.Mean.PM2.5.Concentration: num  1.6 1 1.3 6.3 2.6 1.2 1.5 1.5 3.7 1.6 ...
##  $ UNITS                         : Factor w/ 1 level "ug/m3 LC": 1 1 1 1 1 1 1 1 1 1 ...
##  $ DAILY_AQI_VALUE               : int  7 4 5 26 11 5 6 6 15 7 ...
##  $ Site.Name                     : Factor w/ 25 levels "","Board Of Ed. Bldg.",..: 14 14 14 14 14 14 14 14 14 14 ...
##  $ DAILY_OBS_COUNT               : int  1 1 1 1 1 1 1 1 1 1 ...
##  $ PERCENT_COMPLETE              : num  100 100 100 100 100 100 100 100 100 100 ...
##  $ AQS_PARAMETER_CODE            : int  88502 88502 88502 88502 88502 88502 88502 88502 88502 88502 ...
##  $ AQS_PARAMETER_DESC            : Factor w/ 2 levels "Acceptable PM2.5 AQI & Speciation Mass",..: 1 1 1 1 1 1 1 1 1 1 ...
##  $ CBSA_CODE                     : int  NA NA NA NA NA NA NA NA NA NA ...
##  $ CBSA_NAME                     : Factor w/ 14 levels "","Asheville, NC",..: 1 1 1 1 1 1 1 1 1 1 ...
##  $ STATE_CODE                    : int  37 37 37 37 37 37 37 37 37 37 ...
##  $ STATE                         : Factor w/ 1 level "North Carolina": 1 1 1 1 1 1 1 1 1 1 ...
##  $ COUNTY_CODE                   : int  11 11 11 11 11 11 11 11 11 11 ...
##  $ COUNTY                        : Factor w/ 21 levels "Avery","Buncombe",..: 1 1 1 1 1 1 1 1 1 1 ...
##  $ SITE_LATITUDE                 : num  36 36 36 36 36 ...
##  $ SITE_LONGITUDE                : num  -81.9 -81.9 -81.9 -81.9 -81.9 ...
\end{verbatim}

\begin{Shaded}
\begin{Highlighting}[]
\KeywordTok{summary}\NormalTok{(EPAair_PM25_NC2019)}
\end{Highlighting}
\end{Shaded}

\begin{verbatim}
##          Date         Source        Site.ID               POC       
##  02/26/2019:  41   AirNow:1670   Min.   :370110002   Min.   :1.000  
##  01/21/2019:  40   AQS   :6911   1st Qu.:370630015   1st Qu.:3.000  
##  02/14/2019:  40                 Median :371190041   Median :3.000  
##  01/09/2019:  39                 Mean   :371023743   Mean   :3.032  
##  01/27/2019:  39                 3rd Qu.:371290002   3rd Qu.:3.000  
##  02/02/2019:  39                 Max.   :371830021   Max.   :5.000  
##  (Other)   :8343                                                    
##  Daily.Mean.PM2.5.Concentration      UNITS      DAILY_AQI_VALUE
##  Min.   :-3.100                 ug/m3 LC:8581   Min.   : 0.00  
##  1st Qu.: 4.900                                 1st Qu.:20.00  
##  Median : 7.400                                 Median :31.00  
##  Mean   : 7.684                                 Mean   :31.51  
##  3rd Qu.:10.100                                 3rd Qu.:42.00  
##  Max.   :31.200                                 Max.   :91.00  
##                                                                
##                 Site.Name    DAILY_OBS_COUNT PERCENT_COMPLETE
##  Millbrook School    : 738   Min.   :1       Min.   :100     
##  Garinger High School: 629   1st Qu.:1       1st Qu.:100     
##  Remount             : 573   Median :1       Median :100     
##  Hickory Water Tower : 518   Mean   :1       Mean   :100     
##  Hattie Avenue       : 436   3rd Qu.:1       3rd Qu.:100     
##  Durham Armory       : 431   Max.   :1       Max.   :100     
##  (Other)             :5256                                   
##  AQS_PARAMETER_CODE                              AQS_PARAMETER_DESC
##  Min.   :88101      Acceptable PM2.5 AQI & Speciation Mass:1029    
##  1st Qu.:88101      PM2.5 - Local Conditions              :7552    
##  Median :88101                                                     
##  Mean   :88149                                                     
##  3rd Qu.:88101                                                     
##  Max.   :88502                                                     
##                                                                    
##    CBSA_CODE                                 CBSA_NAME      STATE_CODE
##  Min.   :11700   Raleigh, NC                      :1441   Min.   :37  
##  1st Qu.:19000   Charlotte-Concord-Gastonia, NC-SC:1379   1st Qu.:37  
##  Median :25860   Winston-Salem, NC                :1235   Median :37  
##  Mean   :31099                                    :1058   Mean   :37  
##  3rd Qu.:40580   Hickory-Lenoir-Morganton, NC     : 518   3rd Qu.:37  
##  Max.   :49180   Durham-Chapel Hill, NC           : 431   Max.   :37  
##  NA's   :1058    (Other)                          :2519               
##             STATE       COUNTY_CODE            COUNTY     SITE_LATITUDE  
##  North Carolina:8581   Min.   : 11.0   Mecklenburg:1379   Min.   :34.36  
##                        1st Qu.: 63.0   Wake       :1083   1st Qu.:35.26  
##                        Median :119.0   Forsyth    : 839   Median :35.73  
##                        Mean   :102.4   Catawba    : 518   Mean   :35.63  
##                        3rd Qu.:129.0   Durham     : 431   3rd Qu.:35.91  
##                        Max.   :183.0   Cumberland : 427   Max.   :36.51  
##                                        (Other)    :3904                  
##  SITE_LONGITUDE  
##  Min.   :-83.44  
##  1st Qu.:-80.87  
##  Median :-80.23  
##  Mean   :-79.95  
##  3rd Qu.:-78.57  
##  Max.   :-76.21  
## 
\end{verbatim}

\begin{Shaded}
\begin{Highlighting}[]
\KeywordTok{class}\NormalTok{(EPAair_PM25_NC2019)}
\end{Highlighting}
\end{Shaded}

\begin{verbatim}
## [1] "data.frame"
\end{verbatim}

\hypertarget{wrangle-individual-datasets-to-create-processed-files.}{%
\subsection{Wrangle individual datasets to create processed
files.}\label{wrangle-individual-datasets-to-create-processed-files.}}

\begin{enumerate}
\def\labelenumi{\arabic{enumi}.}
\setcounter{enumi}{2}
\tightlist
\item
  Change date to date
\item
  Select the following columns: Date, DAILY\_AQI\_VALUE, Site.Name,
  AQS\_PARAMETER\_DESC, COUNTY, SITE\_LATITUDE, SITE\_LONGITUDE
\item
  For the PM2.5 datasets, fill all cells in AQS\_PARAMETER\_DESC with
  ``PM2.5'' (all cells in this column should be identical).
\item
  Save all four processed datasets in the Processed folder. Use the same
  file names as the raw files but replace ``raw'' with ``processed''.
\end{enumerate}

\begin{Shaded}
\begin{Highlighting}[]
\CommentTok{#3 }
\NormalTok{EPAair_O3_NC2018}\OperatorTok{$}\NormalTok{Date <-}\StringTok{ }\KeywordTok{as.Date}\NormalTok{(EPAair_O3_NC2018}\OperatorTok{$}\NormalTok{Date, }\DataTypeTok{format =} \StringTok{"%m/%d/%Y"}\NormalTok{)}
\NormalTok{EPAair_O3_NC2019}\OperatorTok{$}\NormalTok{Date <-}\StringTok{ }\KeywordTok{as.Date}\NormalTok{(EPAair_O3_NC2019}\OperatorTok{$}\NormalTok{Date, }\DataTypeTok{format =} \StringTok{"%m/%d/%Y"}\NormalTok{)}
\NormalTok{EPAair_PM25_NC2018}\OperatorTok{$}\NormalTok{Date <-}\StringTok{ }\KeywordTok{as.Date}\NormalTok{(EPAair_PM25_NC2018}\OperatorTok{$}\NormalTok{Date, }\DataTypeTok{format =} \StringTok{"%m/%d/%Y"}\NormalTok{)}
\NormalTok{EPAair_PM25_NC2019}\OperatorTok{$}\NormalTok{Date <-}\StringTok{ }\KeywordTok{as.Date}\NormalTok{(EPAair_PM25_NC2019}\OperatorTok{$}\NormalTok{Date, }\DataTypeTok{format =} \StringTok{"%m/%d/%Y"}\NormalTok{)}

\CommentTok{#4}
\NormalTok{EPAair_O3_NC2018.processed <-}\StringTok{ }
\StringTok{  }\NormalTok{EPAair_O3_NC2018 }\OperatorTok
\StringTok{  }\KeywordTok{select}\NormalTok{(Date, DAILY_AQI_VALUE, Site.Name, AQS_PARAMETER_DESC, COUNTY, SITE_LATITUDE}\OperatorTok{:}\NormalTok{SITE_LONGITUDE)}

\NormalTok{EPAair_O3_NC2019.processed <-}\StringTok{ }
\StringTok{  }\NormalTok{EPAair_O3_NC2019 }\OperatorTok
\StringTok{  }\KeywordTok{select}\NormalTok{(Date, DAILY_AQI_VALUE, Site.Name, AQS_PARAMETER_DESC, COUNTY, SITE_LATITUDE}\OperatorTok{:}\NormalTok{SITE_LONGITUDE)}

\NormalTok{EPAair_PM25_NC2018.processed <-}\StringTok{ }
\StringTok{  }\NormalTok{EPAair_PM25_NC2018 }\OperatorTok
\StringTok{  }\KeywordTok{select}\NormalTok{(Date, DAILY_AQI_VALUE, Site.Name, AQS_PARAMETER_DESC, COUNTY, SITE_LATITUDE}\OperatorTok{:}\NormalTok{SITE_LONGITUDE) }

\NormalTok{EPAair_PM25_NC2019.processed <-}\StringTok{ }
\StringTok{  }\NormalTok{EPAair_PM25_NC2019 }\OperatorTok
\StringTok{  }\KeywordTok{select}\NormalTok{(Date, DAILY_AQI_VALUE, Site.Name, AQS_PARAMETER_DESC, COUNTY, SITE_LATITUDE}\OperatorTok{:}\NormalTok{SITE_LONGITUDE)}

\CommentTok{#5}
\KeywordTok{levels}\NormalTok{(EPAair_PM25_NC2018.processed}\OperatorTok{$}\NormalTok{AQS_PARAMETER_DESC)[}\KeywordTok{levels}\NormalTok{(EPAair_PM25_NC2018.processed}\OperatorTok{$}\NormalTok{AQS_PARAMETER_DESC)}\OperatorTok{==}\StringTok{"Acceptable PM2.5 AQI & Speciation Mass"}\NormalTok{]<-}\StringTok{ "PM2.5"}

\KeywordTok{levels}\NormalTok{(EPAair_PM25_NC2018.processed}\OperatorTok{$}\NormalTok{AQS_PARAMETER_DESC)[}\KeywordTok{levels}\NormalTok{(EPAair_PM25_NC2018.processed}\OperatorTok{$}\NormalTok{AQS_PARAMETER_DESC)}\OperatorTok{==}\StringTok{"PM2.5 - Local Conditions"}\NormalTok{]<-}\StringTok{ "PM2.5"}

\KeywordTok{levels}\NormalTok{(EPAair_PM25_NC2019.processed}\OperatorTok{$}\NormalTok{AQS_PARAMETER_DESC)[}\KeywordTok{levels}\NormalTok{(EPAair_PM25_NC2019.processed}\OperatorTok{$}\NormalTok{AQS_PARAMETER_DESC)}\OperatorTok{==}\StringTok{"Acceptable PM2.5 AQI & Speciation Mass"}\NormalTok{]<-}\StringTok{ "PM2.5"}

\KeywordTok{levels}\NormalTok{(EPAair_PM25_NC2019.processed}\OperatorTok{$}\NormalTok{AQS_PARAMETER_DESC)[}\KeywordTok{levels}\NormalTok{(EPAair_PM25_NC2019.processed}\OperatorTok{$}\NormalTok{AQS_PARAMETER_DESC)}\OperatorTok{==}\StringTok{"PM2.5 - Local Conditions"}\NormalTok{]<-}\StringTok{ "PM2.5"}

\CommentTok{#6}
\KeywordTok{write.csv}\NormalTok{(EPAair_O3_NC2018.processed, }\DataTypeTok{row.names =} \OtherTok{FALSE}\NormalTok{, }
          \DataTypeTok{file =} \StringTok{"./Data/Processed/EPAair_O3_NC2018_Processed.csv"}\NormalTok{)}

\KeywordTok{write.csv}\NormalTok{(EPAair_O3_NC2019.processed, }\DataTypeTok{row.names =} \OtherTok{FALSE}\NormalTok{, }
          \DataTypeTok{file =} \StringTok{"./Data/Processed/EPAair_O3_NC2019_Processed.csv"}\NormalTok{)}

\KeywordTok{write.csv}\NormalTok{(EPAair_PM25_NC2018.processed, }\DataTypeTok{row.names =} \OtherTok{FALSE}\NormalTok{, }
          \DataTypeTok{file =} \StringTok{"./Data/Processed/EPAair_PM25_NC2018.Processed.csv"}\NormalTok{)}

\KeywordTok{write.csv}\NormalTok{(EPAair_PM25_NC2019.processed, }\DataTypeTok{row.names =} \OtherTok{FALSE}\NormalTok{, }
          \DataTypeTok{file =} \StringTok{"./Data/Processed/EPAair_PM25_NC2019.Processed.csv"}\NormalTok{)}
\end{Highlighting}
\end{Shaded}

\hypertarget{combine-datasets}{%
\subsection{Combine datasets}\label{combine-datasets}}

\begin{enumerate}
\def\labelenumi{\arabic{enumi}.}
\setcounter{enumi}{6}
\tightlist
\item
  Combine the four datasets with \texttt{rbind}. Make sure your column
  names are identical prior to running this code.
\item
  Wrangle your new dataset with a pipe function (\%\textgreater{}\%) so
  that it fills the following conditions:
\end{enumerate}

\begin{itemize}
\tightlist
\item
  Include all sites that the four data frames have in common: ``Linville
  Falls'', ``Durham Armory'', ``Leggett'', ``Hattie Avenue'', ``Clemmons
  Middle'', ``Mendenhall School'', ``Frying Pan Mountain'', ``West
  Johnston Co.'', ``Garinger High School'', ``Castle Hayne'', ``Pitt
  Agri. Center'', ``Bryson City'', ``Millbrook School'' (the function
  \texttt{intersect} can figure out common factor levels)
\item
  Some sites have multiple measurements per day. Use the
  split-apply-combine strategy to generate daily means: group by date,
  site, aqs parameter, and county. Take the mean of the AQI value,
  latitude, and longitude.
\item
  Add columns for ``Month'' and ``Year'' by parsing your ``Date'' column
  (hint: \texttt{lubridate} package)
\item
  Hint: the dimensions of this dataset should be 14,752 x 9.
\end{itemize}

\begin{enumerate}
\def\labelenumi{\arabic{enumi}.}
\setcounter{enumi}{8}
\tightlist
\item
  Spread your datasets such that AQI values for ozone and PM2.5 are in
  separate columns. Each location on a specific date should now occupy
  only one row.
\item
  Call up the dimensions of your new tidy dataset.
\item
  Save your processed dataset with the following file name:
  ``EPAair\_O3\_PM25\_NC1718\_Processed.csv''
\end{enumerate}

\begin{Shaded}
\begin{Highlighting}[]
\CommentTok{#7}
\NormalTok{EPAair_O3_PM25_NC <-}\StringTok{ }\KeywordTok{rbind}\NormalTok{(EPAair_O3_NC2018.processed, EPAair_O3_NC2019.processed, EPAair_PM25_NC2018.processed, EPAair_PM25_NC2019.processed)}
\KeywordTok{dim}\NormalTok{(EPAair_O3_PM25_NC)}
\end{Highlighting}
\end{Shaded}

\begin{verbatim}
## [1] 37893     7
\end{verbatim}

\begin{Shaded}
\begin{Highlighting}[]
\CommentTok{#8}
\NormalTok{EPAair_O3_PM25_NC1718 <-}\StringTok{ }
\StringTok{  }\NormalTok{EPAair_O3_PM25_NC }\OperatorTok
\StringTok{  }\KeywordTok{filter}\NormalTok{(Site.Name }\OperatorTok{==}\StringTok{ "Linville Falls"}\OperatorTok{|}\StringTok{ }\NormalTok{Site.Name }\OperatorTok{==}\StringTok{ "Durham Armory"}\OperatorTok{|}\NormalTok{Site.Name }\OperatorTok{==}\StringTok{"Leggett"}\OperatorTok{|}\StringTok{ }\NormalTok{Site.Name }\OperatorTok{==}\StringTok{ "Hattie Avenue"}\OperatorTok{|}\StringTok{ }\NormalTok{Site.Name }\OperatorTok{==}\StringTok{ "Clemmons Middle"} \OperatorTok{|}\StringTok{ }\NormalTok{Site.Name }\OperatorTok{==}\StringTok{ "Mendenhall School"} \OperatorTok{|}\StringTok{ }\NormalTok{Site.Name }\OperatorTok{==}\StringTok{ "Frying Pan Mountain"}\OperatorTok{|}\StringTok{ }\NormalTok{Site.Name }\OperatorTok{==}\StringTok{ "West Johnston Co."}\OperatorTok{|}\StringTok{ }\NormalTok{Site.Name }\OperatorTok{==}\StringTok{  "Garinger High School"}\OperatorTok{|}\StringTok{ }\NormalTok{Site.Name }\OperatorTok{==}\StringTok{  "Castle Hayne"}\OperatorTok{|}\StringTok{ }\NormalTok{Site.Name }\OperatorTok{==}\StringTok{  "Pitt Agri. Center"}\OperatorTok{|}\NormalTok{Site.Name }\OperatorTok{==}\StringTok{  "Bryson City"}\OperatorTok{|}\StringTok{ }\NormalTok{Site.Name }\OperatorTok{==}\StringTok{ "Millbrook School"}\NormalTok{)}\OperatorTok
\StringTok{  }\KeywordTok{group_by}\NormalTok{(Date, Site.Name, AQS_PARAMETER_DESC, COUNTY) }\OperatorTok
\StringTok{  }\KeywordTok{summarise}\NormalTok{(}\DataTypeTok{mean_AQI_value =} \KeywordTok{mean}\NormalTok{(DAILY_AQI_VALUE), }
            \DataTypeTok{meanLat =} \KeywordTok{mean}\NormalTok{(SITE_LATITUDE), }
            \DataTypeTok{meanLong =} \KeywordTok{mean}\NormalTok{(SITE_LONGITUDE)) }\OperatorTok
\StringTok{  }\KeywordTok{mutate}\NormalTok{(}\DataTypeTok{month =} \KeywordTok{month}\NormalTok{(Date))}\OperatorTok
\StringTok{  }\KeywordTok{mutate}\NormalTok{(}\DataTypeTok{year=} \KeywordTok{year}\NormalTok{(Date))}

\KeywordTok{dim}\NormalTok{(EPAair_O3_PM25_NC1718)}
\end{Highlighting}
\end{Shaded}

\begin{verbatim}
## [1] 14752     9
\end{verbatim}

\begin{Shaded}
\begin{Highlighting}[]
\CommentTok{#9}
\NormalTok{EPAair_O3_PM25_NC1718.spread <-}\StringTok{ }\KeywordTok{spread}\NormalTok{(EPAair_O3_PM25_NC1718, AQS_PARAMETER_DESC, mean_AQI_value)}

\CommentTok{#10}
\KeywordTok{dim}\NormalTok{(EPAair_O3_PM25_NC1718.spread)}
\end{Highlighting}
\end{Shaded}

\begin{verbatim}
## [1] 8976    9
\end{verbatim}

\begin{Shaded}
\begin{Highlighting}[]
\CommentTok{#11}
\KeywordTok{write.csv}\NormalTok{(EPAair_O3_PM25_NC1718.spread, }\DataTypeTok{row.names =} \OtherTok{FALSE}\NormalTok{, }
          \DataTypeTok{file =} \StringTok{"./Data/Processed/EPAair_O3_PM25_NC1718_Processed.csv"}\NormalTok{)}
\end{Highlighting}
\end{Shaded}

\hypertarget{generate-summary-tables}{%
\subsection{Generate summary tables}\label{generate-summary-tables}}

\begin{enumerate}
\def\labelenumi{\arabic{enumi}.}
\setcounter{enumi}{11}
\item
  Use the split-apply-combine strategy to generate a summary data frame.
  Data should be grouped by site, month, and year. Generate the mean AQI
  values for ozone and PM2.5 for each group. Then, add a pipe to remove
  instances where a month and year are not available (use the function
  \texttt{drop\_na} in your pipe).
\item
  Call up the dimensions of the summary dataset.
\end{enumerate}

\begin{Shaded}
\begin{Highlighting}[]
\CommentTok{#12a}
\NormalTok{EPAair_O3_PM25_summary <-}\StringTok{ }
\StringTok{  }\NormalTok{EPAair_O3_PM25_NC1718.spread }\OperatorTok
\StringTok{  }\KeywordTok{group_by}\NormalTok{(Site.Name, month, year) }\OperatorTok
\StringTok{  }\KeywordTok{summarise}\NormalTok{(}\DataTypeTok{mean_Ozone =} \KeywordTok{mean}\NormalTok{(Ozone), }
            \DataTypeTok{mean_PM2.5 =} \KeywordTok{mean}\NormalTok{(PM2}\FloatTok{.5}\NormalTok{))}\OperatorTok
\StringTok{  }\KeywordTok{drop_na}\NormalTok{(month)}\OperatorTok
\StringTok{  }\KeywordTok{drop_na}\NormalTok{(year)}
  
\CommentTok{#13}
\KeywordTok{dim}\NormalTok{(EPAair_O3_PM25_summary)}
\end{Highlighting}
\end{Shaded}

\begin{verbatim}
## [1] 308   5
\end{verbatim}

\begin{enumerate}
\def\labelenumi{\arabic{enumi}.}
\setcounter{enumi}{13}
\tightlist
\item
  Why did we use the function \texttt{drop\_na} rather than
  \texttt{na.omit}?
\end{enumerate}

\begin{quote}
Answer: The ``na.omit'' function returns any object with incomplete
cases, but it does not remove the N/As from the dataset. The
``drop\_na'' function allows us to remove items with missing values. We
wanted to remove the instances where a month and year are not available.
\end{quote}

\end{document}
