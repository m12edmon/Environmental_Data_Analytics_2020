\PassOptionsToPackage{unicode=true}{hyperref} % options for packages loaded elsewhere
\PassOptionsToPackage{hyphens}{url}
%
\documentclass[]{article}
\usepackage{lmodern}
\usepackage{amssymb,amsmath}
\usepackage{ifxetex,ifluatex}
\usepackage{fixltx2e} % provides \textsubscript
\ifnum 0\ifxetex 1\fi\ifluatex 1\fi=0 % if pdftex
  \usepackage[T1]{fontenc}
  \usepackage[utf8]{inputenc}
  \usepackage{textcomp} % provides euro and other symbols
\else % if luatex or xelatex
  \usepackage{unicode-math}
  \defaultfontfeatures{Ligatures=TeX,Scale=MatchLowercase}
\fi
% use upquote if available, for straight quotes in verbatim environments
\IfFileExists{upquote.sty}{\usepackage{upquote}}{}
% use microtype if available
\IfFileExists{microtype.sty}{%
\usepackage[]{microtype}
\UseMicrotypeSet[protrusion]{basicmath} % disable protrusion for tt fonts
}{}
\IfFileExists{parskip.sty}{%
\usepackage{parskip}
}{% else
\setlength{\parindent}{0pt}
\setlength{\parskip}{6pt plus 2pt minus 1pt}
}
\usepackage{hyperref}
\hypersetup{
            pdftitle={Assignment 2: Coding Basics},
            pdfauthor={Masha Edmondson},
            pdfborder={0 0 0},
            breaklinks=true}
\urlstyle{same}  % don't use monospace font for urls
\usepackage[margin=2.54cm]{geometry}
\usepackage{color}
\usepackage{fancyvrb}
\newcommand{\VerbBar}{|}
\newcommand{\VERB}{\Verb[commandchars=\\\{\}]}
\DefineVerbatimEnvironment{Highlighting}{Verbatim}{commandchars=\\\{\}}
% Add ',fontsize=\small' for more characters per line
\usepackage{framed}
\definecolor{shadecolor}{RGB}{248,248,248}
\newenvironment{Shaded}{\begin{snugshade}}{\end{snugshade}}
\newcommand{\AlertTok}[1]{\textcolor[rgb]{0.94,0.16,0.16}{#1}}
\newcommand{\AnnotationTok}[1]{\textcolor[rgb]{0.56,0.35,0.01}{\textbf{\textit{#1}}}}
\newcommand{\AttributeTok}[1]{\textcolor[rgb]{0.77,0.63,0.00}{#1}}
\newcommand{\BaseNTok}[1]{\textcolor[rgb]{0.00,0.00,0.81}{#1}}
\newcommand{\BuiltInTok}[1]{#1}
\newcommand{\CharTok}[1]{\textcolor[rgb]{0.31,0.60,0.02}{#1}}
\newcommand{\CommentTok}[1]{\textcolor[rgb]{0.56,0.35,0.01}{\textit{#1}}}
\newcommand{\CommentVarTok}[1]{\textcolor[rgb]{0.56,0.35,0.01}{\textbf{\textit{#1}}}}
\newcommand{\ConstantTok}[1]{\textcolor[rgb]{0.00,0.00,0.00}{#1}}
\newcommand{\ControlFlowTok}[1]{\textcolor[rgb]{0.13,0.29,0.53}{\textbf{#1}}}
\newcommand{\DataTypeTok}[1]{\textcolor[rgb]{0.13,0.29,0.53}{#1}}
\newcommand{\DecValTok}[1]{\textcolor[rgb]{0.00,0.00,0.81}{#1}}
\newcommand{\DocumentationTok}[1]{\textcolor[rgb]{0.56,0.35,0.01}{\textbf{\textit{#1}}}}
\newcommand{\ErrorTok}[1]{\textcolor[rgb]{0.64,0.00,0.00}{\textbf{#1}}}
\newcommand{\ExtensionTok}[1]{#1}
\newcommand{\FloatTok}[1]{\textcolor[rgb]{0.00,0.00,0.81}{#1}}
\newcommand{\FunctionTok}[1]{\textcolor[rgb]{0.00,0.00,0.00}{#1}}
\newcommand{\ImportTok}[1]{#1}
\newcommand{\InformationTok}[1]{\textcolor[rgb]{0.56,0.35,0.01}{\textbf{\textit{#1}}}}
\newcommand{\KeywordTok}[1]{\textcolor[rgb]{0.13,0.29,0.53}{\textbf{#1}}}
\newcommand{\NormalTok}[1]{#1}
\newcommand{\OperatorTok}[1]{\textcolor[rgb]{0.81,0.36,0.00}{\textbf{#1}}}
\newcommand{\OtherTok}[1]{\textcolor[rgb]{0.56,0.35,0.01}{#1}}
\newcommand{\PreprocessorTok}[1]{\textcolor[rgb]{0.56,0.35,0.01}{\textit{#1}}}
\newcommand{\RegionMarkerTok}[1]{#1}
\newcommand{\SpecialCharTok}[1]{\textcolor[rgb]{0.00,0.00,0.00}{#1}}
\newcommand{\SpecialStringTok}[1]{\textcolor[rgb]{0.31,0.60,0.02}{#1}}
\newcommand{\StringTok}[1]{\textcolor[rgb]{0.31,0.60,0.02}{#1}}
\newcommand{\VariableTok}[1]{\textcolor[rgb]{0.00,0.00,0.00}{#1}}
\newcommand{\VerbatimStringTok}[1]{\textcolor[rgb]{0.31,0.60,0.02}{#1}}
\newcommand{\WarningTok}[1]{\textcolor[rgb]{0.56,0.35,0.01}{\textbf{\textit{#1}}}}
\usepackage{graphicx,grffile}
\makeatletter
\def\maxwidth{\ifdim\Gin@nat@width>\linewidth\linewidth\else\Gin@nat@width\fi}
\def\maxheight{\ifdim\Gin@nat@height>\textheight\textheight\else\Gin@nat@height\fi}
\makeatother
% Scale images if necessary, so that they will not overflow the page
% margins by default, and it is still possible to overwrite the defaults
% using explicit options in \includegraphics[width, height, ...]{}
\setkeys{Gin}{width=\maxwidth,height=\maxheight,keepaspectratio}
\setlength{\emergencystretch}{3em}  % prevent overfull lines
\providecommand{\tightlist}{%
  \setlength{\itemsep}{0pt}\setlength{\parskip}{0pt}}
\setcounter{secnumdepth}{0}
% Redefines (sub)paragraphs to behave more like sections
\ifx\paragraph\undefined\else
\let\oldparagraph\paragraph
\renewcommand{\paragraph}[1]{\oldparagraph{#1}\mbox{}}
\fi
\ifx\subparagraph\undefined\else
\let\oldsubparagraph\subparagraph
\renewcommand{\subparagraph}[1]{\oldsubparagraph{#1}\mbox{}}
\fi

% set default figure placement to htbp
\makeatletter
\def\fps@figure{htbp}
\makeatother


\title{Assignment 2: Coding Basics}
\author{Masha Edmondson}
\date{}

\begin{document}
\maketitle

\hypertarget{overview}{%
\subsection{OVERVIEW}\label{overview}}

This exercise accompanies the lessons in Environmental Data Analytics on
coding basics.

\hypertarget{directions}{%
\subsection{Directions}\label{directions}}

\begin{enumerate}
\def\labelenumi{\arabic{enumi}.}
\tightlist
\item
  Change ``Student Name'' on line 3 (above) with your name.
\item
  Work through the steps, \textbf{creating code and output} that fulfill
  each instruction.
\item
  Be sure to \textbf{answer the questions} in this assignment document.
\item
  When you have completed the assignment, \textbf{Knit} the text and
  code into a single PDF file.
\item
  After Knitting, submit the completed exercise (PDF file) to the
  dropbox in Sakai. Add your last name into the file name (e.g.,
  ``Salk\_A02\_CodingBasics.Rmd'') prior to submission.
\end{enumerate}

The completed exercise is due on Tuesday, January 21 at 1:00 pm.

\hypertarget{basics-day-1}{%
\subsection{Basics Day 1}\label{basics-day-1}}

\begin{enumerate}
\def\labelenumi{\arabic{enumi}.}
\item
  Generate a sequence of numbers from one to 100, increasing by fours.
  Assign this sequence a name.
\item
  Compute the mean and median of this sequence.
\item
  Ask R to determine whether the mean is greater than the median.
\item
  Insert comments in your code to describe what you are doing.
\end{enumerate}

\begin{Shaded}
\begin{Highlighting}[]
\CommentTok{#1. }
\NormalTok{   my_sequence <-}\StringTok{ }\KeywordTok{seq}\NormalTok{(}\DecValTok{0}\NormalTok{, }\DecValTok{100}\NormalTok{, }\DataTypeTok{by =} \DecValTok{4}\NormalTok{) }
\CommentTok{#naming the sequence "my sequence" and using the sequence function to create a sequence of numbers from one to 100 increasing by fours.}

\CommentTok{#2. }
   \KeywordTok{mean}\NormalTok{(my_sequence)  }\CommentTok{#finding the mean of the sequence through the mean function}
\end{Highlighting}
\end{Shaded}

\begin{verbatim}
## [1] 50
\end{verbatim}

\begin{Shaded}
\begin{Highlighting}[]
   \KeywordTok{median}\NormalTok{(my_sequence) }\CommentTok{#finding the median of the sequence through the median function}
\end{Highlighting}
\end{Shaded}

\begin{verbatim}
## [1] 50
\end{verbatim}

\begin{Shaded}
\begin{Highlighting}[]
\CommentTok{#3. }
   \KeywordTok{mean}\NormalTok{(my_sequence) }\OperatorTok{>}\StringTok{ }\KeywordTok{median}\NormalTok{(my_sequence) }
\end{Highlighting}
\end{Shaded}

\begin{verbatim}
## [1] FALSE
\end{verbatim}

\begin{Shaded}
\begin{Highlighting}[]
   \CommentTok{#asking a logical statement if the mean of the sequence is greater than the median of the sequence}
\end{Highlighting}
\end{Shaded}

\hypertarget{basics-day-2}{%
\subsection{Basics Day 2}\label{basics-day-2}}

\begin{enumerate}
\def\labelenumi{\arabic{enumi}.}
\setcounter{enumi}{4}
\item
  Create a series of vectors, each with four components, consisting of
  (a) names of students, (b) test scores out of a total 100 points, and
  (c) whether or not they have passed the test (TRUE or FALSE) with a
  passing grade of 50.
\item
  Label each vector with a comment on what type of vector it is.
\item
  Combine each of the vectors into a data frame. Assign the data frame
  an informative name.
\item
  Label the columns of your data frame with informative titles.
\end{enumerate}

\begin{Shaded}
\begin{Highlighting}[]
\NormalTok{students <-}\StringTok{ }\KeywordTok{c}\NormalTok{(}\StringTok{"Abby Brown"}\NormalTok{, }\StringTok{"Joe Fields"}\NormalTok{, }\StringTok{"Eddie Cook"}\NormalTok{, }\StringTok{"Fran Johnson"}\NormalTok{) }\CommentTok{#creating a character vector with student names}

\NormalTok{test_scores <-}\StringTok{ }\KeywordTok{c}\NormalTok{(}\DecValTok{100}\NormalTok{, }\DecValTok{82}\NormalTok{, }\DecValTok{75}\NormalTok{, }\DecValTok{43}\NormalTok{) }\CommentTok{#creating a numeric vector with each student's test scores}

\NormalTok{passed <-}\StringTok{ }\NormalTok{test_scores }\OperatorTok{>=}\StringTok{ }\DecValTok{50} \CommentTok{#creating a logical vector showing which students passed the exam with a score of 50 or higher, and which students faile with a score lower than 50.}

\NormalTok{test_results <-}\StringTok{ }\KeywordTok{data.frame}\NormalTok{(students, test_scores, passed) }\CommentTok{#combining vectors into a dataframe and naming the dataframe}

\KeywordTok{names}\NormalTok{(test_results) <-}\StringTok{ }\KeywordTok{c}\NormalTok{(}\StringTok{"Student"}\NormalTok{,}\StringTok{"Test Score"}\NormalTok{,}\StringTok{"Passed Exam"}\NormalTok{); }\KeywordTok{View}\NormalTok{(test_results) }\CommentTok{#labeling the columns and viewing the data frame created}
\end{Highlighting}
\end{Shaded}

\begin{enumerate}
\def\labelenumi{\arabic{enumi}.}
\setcounter{enumi}{8}
\tightlist
\item
  QUESTION: How is this data frame different from a matrix?
\end{enumerate}

\begin{quote}
Answer: In a data frame, the columns can contain different types of data
(numeric, logical, and character statements), but in a matrix all the
elements are the same type of data which are usually numbers. Data
frames can also combine features of matrices and lists, and the items of
the list serve as the columns of the data frame.
\end{quote}

\begin{enumerate}
\def\labelenumi{\arabic{enumi}.}
\setcounter{enumi}{9}
\item
  Create a function with an if/else statement. Your function should
  determine whether a test score is a passing grade of 50 or above (TRUE
  or FALSE). You will need to choose either the \texttt{if} and
  \texttt{else} statements or the \texttt{ifelse} statement. Hint: Use
  \texttt{print}, not \texttt{return}. The name of your function should
  be informative.
\item
  Apply your function to the vector with test scores that you created in
  number 5.
\end{enumerate}

\begin{Shaded}
\begin{Highlighting}[]
\CommentTok{#test_scores <- c(100, 82, 75, 43) }
\CommentTok{#for (value in test_scores)\{print(value)\}}
\CommentTok{#for (value in test_scores)\{}
  \CommentTok{#if (test_scores >= 50)\{print("Passed Exam")\} else \{print("Failed Exam")\} }
\CommentTok{#tried to create and "if" and "else" statement function for the test score results, which did not end up working.}

\NormalTok{test_results2 <-}\StringTok{ }\KeywordTok{ifelse}\NormalTok{(test_scores }\OperatorTok{>=}\StringTok{ }\DecValTok{50}\NormalTok{, }\StringTok{"passed exam"}\NormalTok{, }\StringTok{"failed exam"}\NormalTok{) }\CommentTok{#created "ifelse" function to determine who passed and who failed the exam given the condition of passing to be a grade of 50 points or higher}
\NormalTok{test_results2 }\CommentTok{#named the outcome of ifelse function}
\end{Highlighting}
\end{Shaded}

\begin{verbatim}
## [1] "passed exam" "passed exam" "passed exam" "failed exam"
\end{verbatim}

\begin{enumerate}
\def\labelenumi{\arabic{enumi}.}
\setcounter{enumi}{11}
\tightlist
\item
  QUESTION: Which option of \texttt{if} and \texttt{else} vs.
  \texttt{ifelse} worked? Why?
\end{enumerate}

\begin{quote}
Answer: When I tried to use the ``if'' and ``else'' statement function
the code failed to run because the condition has a length that was
greater than 1 and only the first element was used- the first test
score. This failed because we were trying to use a vector on the
function when the if, else statement would only give you a response to
one element and not each of the test scores. The ``ifelse'' function
worked because the ``ifsele'' function returns a value with the same
shape as test, an object which can be coerced to logical mode, which is
filled with elements selected from either yes (retun value for true
elements) or no (return values for false elements) depending on whether
the element of test is true or false. This function was able to take our
vector data with four elements and produce a logical statement for each
of the four elements to determine which students passed or failed the
exam.
\end{quote}

\end{document}
